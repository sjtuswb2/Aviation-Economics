\chapter{飞机经济性设计}\label{design2cost}


经济性指标始终是飞机设计过程中需要考虑的重要因素之一,通过开展项目、产品或服务的技术经济分析,做出最优的决策,一直是实现费效比(Cost-Effective Analysis)提升的关键途径。罗列各种不同的经济性指标:飞机的价格,飞机的研发费用,飞机运营成本,飞机的油耗,飞机运营票价和收入,全寿命成本

\section{飞机经济性设计方法}
采用不同的设计指标可以得到不同的设计方案,这些方案的侧重点不同,使用其他经济性指标进行衡量时就未必能够体系出最优的性能。因此,如何确定最优的设计指标,就成为问题的关键。同时,飞机设计过程中存在的大量技术指标,以及目前越来越受到重视的环保指标,使得问题更加复杂化,需要采用多学科、多目标、多约束的全局和局部相耦合的、多层次的优化方法开展优化设计,服务于决策系统。存在的问题:设计决策的层次化,碎片化;数据量和信息量的增加;决策需求和信息供给失衡问题;对数据的分析方法对经济学、管理学和博弈理论的应用,以及和工程技术领域,信息技术领域的深度融合。

\subsection{基于经济性的飞机总体技术方案设计}
从飞机项目的历史经验来看,在飞机方案设计阶段及初步设计早期的设计决策锁定了飞机全寿命成本的$85\%$以上,也基本决定了飞机的直接使用成本,进而影响了飞机的市场竞争力,因此在设计初期开展不同设计方案以及不同设计参数的经济性分析和优化能够起到事半功倍的效果,也是提高飞机市场竞争力的关键环节。实现分析飞机总体方案中主要技术参数与全寿命周期经济性指标(研发成本、单机成本和直接使用成本等)的关联,提供多技术方案决策时的经济性优化工具。民用飞机经济性设计方法,是将经济性指标作为飞机概念设计综合迭代优化过程中的目标函数之一,与安全性、性能、环保、可靠性、舒适性等指标同时考虑的设计方法。

民用飞机经济性设计的基础在于成本估算,飞机研制中为控制飞机项目的研制成本,降低飞机产品的运营费用,尽管成本的估算方法由来已久,但是需要结合我国制造商实际需求,采用适用的经济性指标。在经济性设计方法中,设计者可以自行通过有机融入到概念设计中的成本模型对设计方案进行经济性评估,并能快速响应设计方案,使设计者能将成本与气动、性能、环保等因素同时考虑。

估算模型的建立,更重要的目的是将成本模型有机融入到概念设计过程中,使得概念设计的设计参数作为成本模型的输入端,提高成本模型对设计参数的敏感性。例如使用飞机的性能数据计算飞行轮挡时间,作为直接运营成本的输入,能很好的增加运营成本对设计参数的敏感性。
为得到经济性良好的设计方案,将经济性分析应用于飞机优化设计过程中,首先需要区分飞机总体设计参数的类型以及其参数范围,例如,对于像飞机的设计航程,飞机座位数,单发升限以及机场的起飞降落距离、进场速度等描述飞机市场适应性的总体设计参数,由于它们是飞机制造商根据目标市场而定的,不需要参与设计迭代。而其他类型的飞机总体方案参数,则是在以经济性指标为目标函数的迭代优化中重点关注的参数。面向经济性的民用飞机总体方案的优化方法的一般流程如下。

首先确定基准机型的总体参数,包括重量、气动和性能等。基准机型应该满足目标市场的设计需求,同时基准机型参数的准确度也很重要,不准确的基准机型参数将会使得总体设计的参数优化产生错误。

随后利用民用飞机经济性设计程序,在满足设计需求的条件下,选择基准机型的几何参数和气动参数(机翼展弦比、机翼面积、机身长细比、增升装置等)、动力参数(发动机推力、发动机单位耗油率等)、设计速度(失速速度等)参数等进行敏感性分析,得出这些总体参数的变化带来的飞机重量、发动机推力需求和轮挡性能等设计参数的变化。
最后,利用经济性估算模型,对参数变化后的设计机型进行经济敏感性分析。以经济性最优为基本原则来选取各项总体设计参数。
在飞机总体方案的设计优化中,以不同的经济性指标为目标函数进行优化都得到了应用,常用的指标包括单机成本,直接使用成本,以及考虑飞机市场价值在内的飞机净现值等,对于商用飞机而言,虽然直接使用成本是客户关心的主要经济性指标,也是飞机设计中的重点考虑的指标,但是,也不能完全忽略其他指标,例如,为了提高飞机的使用经济性而在新技术上的过分投入一方面增加项目风险,另一方面又增加了研发投入,在市场竞争定价的商用飞机市场环境中,大大提高了制造商的财务风险。因此如何平衡制造商和运营商的利益,同时取得市场竞争并非一个简答的问题,需要探索以直接使用成本为主,多种经济性指标相结合的综合应用,以及引入价值链概念,考虑飞机价值随时间动态变化的趋势,综合飞机研制进度影响等内容。这些因素大大增加了飞机概念经济性设计的难度和不确定性。

\subsection{基于经济性的机体结构经济性分析和设计方法研究}
飞机的全寿命周期成本体系中,飞机机体结构的材料和制造成本占据重要的比例,同时还通过影响飞机的重量对飞机的燃油效率和直接使用成本产生影响,随着新材料使用比例的不断增加,如何做出最优的技术经济决策成为一个设计师在飞机总体层次和部件层次面临的问题,需要有效可信的分析方法的支撑。

飞机机体结构的成本估算模型主要包括研发成本和制造成本,服务于飞机全机研制成本和单机成本的估算,对于机体结构部件而言,目前最常用的成本估算方法有三类:工程估算法、参数估算法和类比法,三种方法各有优劣,需要的输入信息量不同,得到结果的准确性也不同。
工程估算法是利用工作包分解自下而上地开展费用估算。由于使用基于物理模型的估算,该方法估算准确度高,但模型涉及的参数众多,信息量大,复杂度高,建模和更改的时间都较长。由于依赖于较多的几何、材料等细节数据,一般该方法都与CAD建模系统紧密融合。

参数估算法是利用汇集起来的用于具有类似用途的飞机的大量现有费用数据,只采用少量的特征参数(如重量、速度等)回归出的一组费用-特征量之间的线性关系式。参数估算法主要应用于项目的初期阶段,在多种方案比较时,能快速及时地为选型决策提供可靠的依据,但是这种方法严重依赖于基础数据的可靠性。需要对数据的长期积累和细致分析。基于参数法和物理仿真模型的费用模型代表了未来发展的方向。
类比法是建立在于过去类似的工程项目进行比较,并根据经验加上修正而得出费用估计。类比法用于旧机改进改型项目估算较为准确,在项目的研制初期,通常作为一种粗略的辅助性方法。

\section{飞机经济性设计体系
}在系统整理飞机的经济性数据的基础上,可以建立完整、灵活的飞机经济性数据体系以及基于飞机经济性分析方法基础上的面向飞机经济性对设计方法以及工具,以满足飞机设计不同阶段的需求,概念设计阶段需要快速并相对准确的经济性指标估算方法,详细设计阶段对成本的估算需要准确可靠,并形成完善的数据积累模式和结果,为经济性设计方法的不断完善提供支撑。
在飞机概念设计阶段对整个项目的经济性影响巨大,需要开展大量设计方案的技术经济性分析;而随着设计的深入,需要处理设计细节对经济性的影响分析,对经济性指标的计算需要达到较高的精度和可靠性;在设计的不同层次,不同的决策人员对项目的决策的影响评估,不同部门的经济性数据的整合、校正、归档和利用也将大大影响整个经济性工作的影响力和可持续性。
可以针对有关的技术数据、经济性数据使用不同的分析方法和结果应用途径


\section{结论}
本报告首先综述目前增升装置CFD分析和设计的主要方法,难点和发展趋势。随后分析了增升装置设计应该遵循的主要原则,设计过程和设计环节中的主要工具的使用。本报告对高效率的增升装置的设计具有一定的参考价值。