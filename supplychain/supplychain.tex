\chapter{Aerospace Supply Chain}\label{ch-supply}

Just as in any other industry, supply chain and its management is a critical issue and a reflection of the industrial competitiveness. The aerospace supply chain has its unique features such as its multi-tier structure, strong integration of defense sector and civil business, high barriers of entry, high demand for capital and technology, etc. Aerospace supply chain is also strongly affected by factors of geopolitics.

The status of aerospace supply chain has evolved over the last four decades due to globalization and outsourcing, rapid development of Chinese economy and industrial capabilities. The fundamental driving factor for supply chain evolution is the transfer of value creation activities. Such transfer maybe prompted by factors such as low cost manufacturing, closer to market, tax or tariff avoidance. 

\section{Introduction}
Design, development, manufacturing, and after-sale services often involves large number of different companies and partners in either a simple or a complicated process of information and physical flows.  Commercial off-the shelf (COTS) or commercially available off-the-shelf products are packaged solutions which can then be adapted to satisfy the needs of the purchasing organization, or integrated into a bigger system, rather than the commissioning of custom-made, or bespoke, solutions. COTS provides alternative to custom-made solutions in terms of lower cost and wider availability. 


\section{Concept of Supply Chain and its Management}




It can be expected that the performances of different suppliers are different, the question here is what are the possible criteria that can be used in the choice and evaluation of suppliers. 



\section{Aerospace Supply Chain}
Over the last hundred years, and the last four decades in particular, the aerospace industry has evolved greatly. Such changes can be found in technology base, business models, design methods and tools. 
