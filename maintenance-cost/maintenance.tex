
\chapter{飞机维修成本}\label{ch2}

\section*{引言}


在飞机的长寿命周期中,飞机的维护状态对飞机的性能和价值具有重要的影响。飞机维护成本在飞机使用成本中占重要的比例,是直接使用成本(Direct Operating  Cost,DOC)中的重要成本项,也是全寿命周期成本的组成部分,是关键的飞机的竞争力指标之一。飞机的维修成本和飞机的设计有关,例如系统的复杂性,维护工作的难度体现在所消耗的时间和材料,使用的工具等方面因素。

飞机的维修成本包括飞机机体结构和系统的维护,以及发动机的维护。维护成本的估算在飞机方案设计以及全寿命周期都非常重要,在方案设计阶段对维修成本的考虑将影响不同技术方案的选择,参数的优化,以及对飞机最终的经济竞争力具有长期的影响,在飞机的全寿命周期中,维修成本是实际发生的使用成本的重要部分。

飞机的维护成本受到很多因素的影响,既包括飞机方案设计决策,也包括飞机的使用环境、使用方式,和维护活动。因此,不同机型和不同航空公司的维护成本差异较大,数据的离散度较高,对飞机在使用寿命周期中的价值产生很大的影响,是飞机价值评估中需要考虑的重要因素之一。

对飞机维护成本的估算采用的方法和其他成本项目的估算类似,通过针对维护成本数据与成本驱动因子的关联分析,建立成本估算公式(Cost Estimation Relationship,CER)是常用的方法之一。这种方法的优势在于可以通过成本驱动因子的识别和拟合关系的确立,分析不同影响因素对成本的潜在影响,服务于设计和运营决策。

\section{飞机维护成本}

飞机的维护成本估算包括了直接维护成本(Direct Maintenance Cost, DMC)和管理支出项目,前者和飞机的维护活动直接相关,后者主要取决于航空公司的一些管理方式。大部分的维护成本模型中将两者综合在一起,主要原因在于航空公司在对维护成本的统计汇总将两者并没有明确的分离。直接维护成本(DMC) 的估算存在较大挑战,数据的离散度较大是重要原因之一。

文献\cite{pearlman1966maintenance}中建立了飞机子系统的维护成本的参数模型,其中对成本驱动因子的选择在后续一些针对成本估算的研究中得到广泛使用。文献\cite{fioriti2018cost}中采用IATA的MCTG(Maintenance Cost Technical Group)\cite{mctg}维护成本数据对维护成本的估算公式的驱动因子和拟合得到的估算公式的系数进行了更新。该方法的主要不足在于以下两个方面:

\begin{itemize}
    \item 对于维护成本和驱动因子之间采用了线性假设关系;
    \item 发动机维护成本过于简单,仅仅考虑的发动机数据和发动机推力数据。
\end{itemize}

由于发动机维护成本将是下一节讨论的主要内容,此处重点讨论飞机机体和系统的维护成本的估算方法。这也是主要的维护成本模型中采用的维护成本的分解方法-机体维护成本和发动机维护成本,其中前者包含了结构部件和系统。

在维修成本数据的统计中,一般针对维修活动类型进行,例如航线维修和基地维修,以及部件维修,和发动机维修,其中部件维修一般涉及到的是针对部件和系统供应商开展的维修活动,

典型的维护成本模型包括ATA方法,AEA方法,

在常用的飞机维护成本估算公式

\section{发动机维护成本}


\section{系统维护成本}



\section{小结}
