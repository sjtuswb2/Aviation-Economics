\chapter{Value Driven Design}\label{chvdd}

In previous chapters, a number of economic merits have been presented and discussed for its use in the economic evaluation of commercial aircraft and design projects. For example, manufactures tend to use net present value to evaluate the likely break-even point, on the other hand, airlines tend to use direct operating cost in decisions on aircraft purchases. Increase in research and development cost by manufacturers would generally lead to lower operating cost by reducing fuel consumption due to lower aerodynamic drag, structural weight, and lower system maintenance cost etc. When the aircraft sales price is under pressure in a market condition, the manufacturers would face more challenges in achieving financial balances as the increased project could not be passed onto airlines. In market reality, a dynamic balance would be reached, in which, the number of better aircraft would benefit and gradually increase its market share. The total value of the industry is increasing. 

In daily life, we might have different understandings on the concept of value, for example, the value provided a phone has changed from the traditional long-distance talking to video calling and many other applications on the smart phone. The importance of the original functionalities have decreased. This changed the way and merits we evaluate the product changes over time. On the other hand, functionalities provided by other products remained largely the same, but the performance and other aspects simply enhanced its main, original functionality. In both scenarios, the key for sustained success comes from its value creation nature for players from each category along the supply chain. The change in supply chain landscape is undoubtedly driven by fundamentals in the such value creation dynamics. 

Back to aerospace industry, here is a few examples of commercial aircraft development project and defense projects. The focus is on the economic performance. There is no doubt that the reasons for such delays and over budget are very complex including technical challenges, policies, economic conditions, management, etc. Due to its important impact on the project, company and sometimes on the health of the whole industry, significant effort have been devoted to this challenge. Value-driven design is one such methodology. 







