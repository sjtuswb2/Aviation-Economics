\chapter{概述}\label{ch1}

%\ding{172}=1 in circle
%\chapter{}\label{ch1}


民用航空运输业是全球经济的重要组成部分,与之相关联的飞机制造业不仅属于高科技制造业,同时是全球化程度很高的产业,航空航天产业也和国防需求密切相关,是国家制造业全球竞争力的重要体现。

飞机设计单位有一句话:“{\textbf{飞机设计,气动先行}}”。这并不是说在飞机设计中其它的专业不重要,飞机设计有几十个专业,都是必不可缺的。而是说,飞机设计首先必须考虑空气动力学问题。

本人遇到过不少航空爱好者,都是一些非航空专业的人士,有工人、农民、有在校学生、还有大学教授。他们设计的飞机,有战斗机、旅客机,水陆两栖的、还有海陆空三用的。他们花了许多的时间和精力,做了大量的研究设计工作。但是,一个共同的特点是没有首先考虑空气动力学问题,自己认为设计的飞机比现在天上飞的飞机都特别、都先进,其实,飞机能否上天飞行都是个问题。

在飞机设计单位还有一句话:``{\textbf{搞气动的人走路都带攻角}}"。意思是气动设计人员很骄傲, 走路都抬着头,目中无人。这种骄傲的思想当然不好,但另一方面反映气动工作在飞机设计中的重要地位。

\index{空气动力学}空气动力学怎么先行?又怎么重要? 这就是本章一开始需要阐述的主题——《{\textbf{空气动力学在飞机设计中的作用}}》。然后介绍一些与本书后面几章有关的、型号空气动力学的基本知识: 主要大气参数和飞机的气动特性。

\section{飞机的空气动力学设计}
\subsection{什么叫``空气动力学"}

